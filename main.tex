%% start of file `template.tex'.
%% Copyright 2006-2010 Xavier Danaux (xdanaux@gmail.com).
%
% This work may be distributed and/or modified under the
% conditions of the LaTeX Project Public License version 1.3c,
% available at http://www.latex-project.org/lppl/.


\documentclass[11pt,a4paper]{moderncv}
\moderncvtheme[blue]{classic}  % optional argument are 'blue'
%(default), 'orange', 'red', 'green', 'grey' and 'roman' (for roman fonts, instead of sans serif fonts)
\usepackage[scale=0.8]{geometry} % adjust the page margins

\firstname{Alexandre\\}\familyname{Kandalintsev\vspace{1.5em}}
%\title{CURRICULUM VIT\AE{}}
%\extrainfo{16/06/1985, Belarusian}
\address{}{Trento, Italy}
\mobile{+39 389 532 69 80}
\email{exe.sre@gmail.com}
\extrainfo{
\homepagesymbol\httplink{uk.linkedin.com/in/exesre} \\
\homepagesymbol\httplink{github.com/kopchik} \\
\vspace{3em}
}
\photo[100pt]{face}


\nopagenumbers{}
\usepackage{multibib}
\newcites{book,misc}{{Books},{Others}}

\begin{document}
\maketitle

\vspace{-30mm}
\section{Work Experience}

\cventry{June 2016 -- now}{One Man Army}{AdminCup (no site yet)}{}{}{
I always dreamed about having a nice pet project and here the dream come true :).
It's an e-learning system for system administrators.
The first demo is yet on its way, but I can summarize the technologies I'm using:
\begin{itemize}
\item PostgreSQL + SQLAlchemy for database
\item ZeroMQ for RPC
\item aiohttp (Python) + nginx for backend
\item Ansible for deployment / automation
\item QEMU / LXD for isolation and containers
\end{itemize}
}

\cventry{Nov 2015 -- now}{Infrastructure Architect, Backend Developer (Python)}{\href{codesign.io}{CoDesign.IO}}{}{}{
I have two roles in this project: software developer and infrastructure architect.
As a developer, I work on our Django backend (Python).
As an infrastructure guy I do ... well, pretty much everything required to run our production platform :).
My duties and contribution:
\begin{itemize}
\item \textbf{Backend:} database query/layout optimization, caching, code sanitizing (pyflakes, autopep8), tests (py.test)
\item \textbf{Infrastructure:} migration to private infrastructure, backups, monitoring, infrastructure-as-a-code (saltstack)
\item \textbf{General duties}: architecture design and implementation, documentation, code review, supervision of interns, customer support.
\end{itemize}
}

\cventry{Jul 2014 -- Jan 2015}{Site Reliability Engineer}{\href{http://goo.gl/Nx9qXo}{Google}}{London, UK}{}{
Support of AdWords/Double-click backends. Some highlights:
\begin{itemize}
\item Developed team-wide Service Level Objectives (SLO)
  \begin{itemize}
  \item Defined objectives for supported projects
  \item Created general guidelines for new projects
  \item Designed and implemented SLO dashboard (\emph{Python} and \emph{JS/JQuery/NVD3})
  \end{itemize}
\item Implemented script that exported job parameters to team dashboard (\emph{Python})
  \begin{itemize}
  \item Queried database for running and recently terminated job
  \item Parsed config files for periodic ("cron") jobs
  \item Performed sanity checks: permission problems, duplicate jobs and other checks
  \end{itemize}
\item {Maintained high SLO}
  \begin{itemize}
  \item Configured monitoring and alerting systems
  \item Resolved production issues
  \end{itemize}
\item General duties:
  \begin{itemize}
  \item Standardization and unification of production configuration
  \item Review of proposed system designs
  \item Performance analysis
  % \item System tunning
  \end{itemize}
\end{itemize}
}

% \cventry{May 2013 -- October 2014}{Internship}{University of Luxembourg}{Luxembourg}{\texttt{http://uni.lu/snt}}{
% % Supervisor: Dzmitry Kliazovich (\texttt{http://disi.unitn.it/~klezovic/})
% \begin{itemize}
% \item {Studied and proposed solutions to reduce the performance interference between VMs}
% \item {The obtained results were submitted as a conference paper}
% \end{itemize}
% }

% \cventry{Sep 2010 -- Nov 2010}{Internship}{University of Trento}{Italy}{\texttt{http://www.napa-wine.eu/}}{
% % Supervisor: Renato Lo Cigno (\texttt{http://disi.unitn.it/locigno/})
% NAPA-WINE  is a peer-to-peer media streaming platform. My contribution is:
% \begin{itemize}
% \item{Support for multi-channel media streams}
% \item{New GUI frontend}
% \end{itemize}
% }

\cventry{Nov 2007 -- Jul 2010}{Senior Unix Administrator}{Agava Company}{Moscow, Russia}{}{
\begin{itemize}
\item{Managed a team of four system administrators}
\item{Solved the most complicated technical problems}
\item{Consulted VIP-clients and provided them with personal support}
\item {Gave trainings and seminars for system administrators on the following topics:}
\begin{itemize}
% \item Apache web server course
\item Understanding Nginx and ``10k clients'' problem
\item Using tcpdump to watch traffic
% \item DDoS defense methods
\item Basic database tunning and optimization
\item Right backup strategies
\end{itemize}
\end{itemize}
\vspace{3mm}
}

\cventry{Oct 2005 -- Nov 2007}{Unix Administrator}{Agava company}{Moscow, Russia}{}{
Standard requirements and duties of a system administrator.
}

% \cventry{Feb 2005 -- Oct 2005}{Junior Unix Administrator}{Agava Company}{Moscow, Russia}{}{
% Provided the first line full-time technical support.
% Assembled, installed and maintained equipment in data centers.
% }


\section{Education}
\cventry{2010 -- 2016}{PhD}{Telecommunications}{University of Trento}{Italy}{Thesis: ``Application Interference in Multi-Core Architectures: Analysis and Effects''}{}
\cventry{2006 -- 2010}{MSc}{Computer Science}{University of Russian Academy of Education}{Moscow, Russia}{}{}
% \cventry{2003 -- 2005}{BSs (3 semesters)}{Computer Systems and Networks}{Bauman Moscow State Technical University}{Moscow, Russia}{}{}


\section{Selected Personal Projects}
\cvitem{}{\textbf{Qemu named interfaces:}
  Assign meaningful names to tap interfaces:
  \url{https://github.com/kopchik/patches/blob/master/qemu_ifname.patch}}
\cvitem{}{\textbf{Virtual Machine Commander}: a tool to orchestrate multiple virtual machines at once:
\url{github.com/kopchik/vmc}}
\cvitem{}{\textbf{Parallel Downloader:} downloads chunks in order so video files can be viewed before they completely downloaded: \url{github.com/kopchik/pdl}}
\cvitem{}{\textbf{IPv4/IPv6 proxy:} enables IPv6 for legacy applications: \url{github.com/kopchik/proxy64}}

\section{Publications}
\nocite{*}
\renewcommand*{\bibliographyhead}[1]{}
\bibliographystyle{abbrv}
\bibliography{main}


%\newpage
\section{Skills}
% \cvitem{OS}{Linux, FreeBSD}{}{}
\cvitem{Programming}{
\begin{description}
\item[Advanced:] Python
\item[Intermediate:] shell scripting
\item[Basic skills:] SQL, Java, JavaScript/LiveScript, C, C++
\end{description}}{}{}
\vspace{-1em}
\cvitem{Tools}{Valgrind, GDB, Qemu/KVM, Docker/LXC/LXD, perf, SaltStack, OpenVpn, VCS and other tools}{}{}

\section{Languages}
\cvlanguage{English}{fluent (C1)}{}
% \cvlanguage{Italian}{elementary (A2)}{}
\cvlanguage{Russian}{native}{}

% \section{Research Interests}
% \cvitem{}{Clouds, resource management, distributed systems.}
\section{Hobbies}
\cvline{}{\small DIY electronics (STM32 microcontrollers, robot platforms, linear power supplies, interaction with the real world).}


\end{document}
